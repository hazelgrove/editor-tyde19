\documentclass[format=sigplan,dvipsnames,backend=bibtex]{acmart}

%%% Top Matter %%%
\title{Extended Abstract: Flexible Structure Editing of Well-Typed Expressions}

\author{David Moon}
\affiliation{University of Colorado Boulder}
\email{dmoon1221@gmail.com}

\author{Cyrus Omar}
\affiliation{University of Chicago}
\email{cyrus.omar@gmail.com}

\author{R.~Benjamin Shapiro}
\affiliation{University of Colorado Boulder}
\email{ben.shapiro@colorado.edu}

\acmConference{TyDe’19}{August 18, 2019}{Berlin, Germany}
\acmISBN{}
\acmDOI{}

\setcopyright{none}

\settopmatter{printacmref=false}

\bibliographystyle{ACM-Reference-Format}
%\citestyle{acmauthoryear}
%%%%%%


%%% Packages + Commands %%%
\usepackage{microtype}
\usepackage{mdframed}
\usepackage{mathpartir}
\usepackage{enumitem}
\usepackage{bbm}
\usepackage{stmaryrd}
\usepackage{mathtools}
\usepackage{leftidx}
\usepackage{wrapfig}
\usepackage{extarrows}
\usepackage{xspace}
\usepackage{todonotes}
\usepackage{bm}
\usepackage{xcolor}
\usepackage{enumitem}

% https://tex.stackexchange.com/a/8633
\usepackage{float}

% inconsolata font
\usepackage{zi4}

\input{commands}
\input{macros}

\newcommand{\Hazel}{\textsf{Hazel}\xspace}

\setlength{\fboxsep}{1.5pt}
\newcommand{\key}[1]{\fbox{\texttt{#1}}}

\begin{document}
\maketitle

\section{Introduction}

Structure editors allow programmers to edit the tree structure of a program directly.
They can provide cognitive benefits for novice programmers, simplify language composition,
  and improve the availability of editor services.
They are also known for being hard to use.

We present the structure editor design of \Hazel, a live functional programming language
	and editor.
By virtue of its type-aware edit actions and execution semantics for incomplete programs, 
	\Hazel enforces the unique invariant that every program edit state is not only syntactically 
	well-formed, but also statically \cite{Hazelnut} and dynamically \cite{HazelnutLive} well-defined.
Central to these guarantees is the use of \emph{typed holes} to enclose incomplete and
	type-inconsistent parts of the program.

Prior structure editors have attempted to improve usability by allowing tightly scoped
	violations of structural invariants.
For example, the Cornell Program Synthesizer enforces high-level tree structure but allows
	the programmer to construct expressions and assignment statements via text editing \cite{Cornell}.
\Hazel takes a different approach: it loosens its structural invariants by formally
	modeling incomplete programs.
While this helps in some ways with designing for usability, \Hazel's semantic guarantees
	require strict adherence to these invariants at all times.
We present our attempt at designing an ergonomic structure editor under these constraints.
	
\section{Structure Editor Design}

We designed \Hazel's structure
	editor to maximize carryover of users' text-editing intuitions.
We motivate four main features supporting \Hazel's ergonomics, then demonstrate them
	in a concrete example.

\parahead{\emph{(1)} Automatic Hole Insertion}
%\todo{maybe add something about how other structure editors do syntactic holes}
Naïvely, enforcing an injunction on ill-typed edit states would force programmers to
	construct programs in a rigid ``outside-in'' manner.
\Hazel addresses this problem by automatically enclosing unfinished or type-inconsistent
	parts of the program in holes.
By internalizing type inconsistencies into its semantics, \Hazel allows for natural
	left-to-right construction of programs while maintaining well-typedness.

\parahead{\emph{(2)} Tree Flattening}
While plain text is a linear sequence of characters, text editors present it in a
	two-dimensional interface by dividing the sequence into rows.
\Hazel's syntax explicitly encodes the vertical and horizontal linearity of text
	editing to which computer users are accustomed.
Expressions in \Hazel are broken up into sequences of line items; 
	within each line item, sequences of infix operators are encoded sequentially.
As we will see later, line items and operands serve as useful landmarks for complex
	node transformations.

\parahead{\emph{(3)} Explicit Tree Signifiers}
Early structure editors were hard to use because they presented a textual notation
	but had editing interfaces requiring awareness of the underlying tree structure.
Comtemporary tools have significantly improved usability, but issues remain.
In a controlled user evaluation of MPS, a state-of-the-art structure editor,
	Berger \emph{et al.}~ \cite{ProjEfficiency}
	report that novice users perceive selection as inaccurate relative to that in text
	editing, and that both novices \emph{and experts} perceive deletion as relatively inaccurate.
These results suggest that there remain discrepancies between the visual signifiers and
	editing affordances.

To address this problem, \Hazel features a novel cursor system that augments the familiar
	cursor of text editors with visual markers of the current node's tree structure.
Not only do these markers faciliate understanding of the program's tree structure,
	but they also clearly indicate what would be removed by deletion.

\parahead{\emph{(4)} Node Staging Mode}
Prior work on structure editing ergonomics has proposed a variety of solutions to 
	constructing infix operator sequences in the manner of text editing, i.e., with the
	same keystrokes.
There has been no similar attention paid to complex tree transformations involving other 
	syntactic forms.
\Hazel features a novel \emph{node staging mode} that faciliates exploration of valid
	node transformations.
Whereas other structure editors require a ``configure then invoke'' flow, where child
	nodes must be selected before invoking construction of the new parent node, \Hazel's
	node staging mode enables a more natural ``invoke then configure'' flow, similar to
	that of text editing.

\subsection{Example}

\newcommand{\feat}[1]{{\bfseries (#1)}}

We now present an example-driven overview of features \feat{2}, \feat{3},
	and \feat{4}, deferring to \cite{Hazelnut} for examples of \feat{1}.

Suppose we are implementing a combat game and, specifically, defining a function
	\texttt{damage : Attack -> Num}. An \texttt{Attack} is a tuple consisting of the
	attack type and a critical hit multiplier, and the returned \texttt{Num} is the
	damage points inflicted upon the current player.
	
{\centering
  \includegraphics[width=\linewidth]{fig/context.png}\par
}
\noindent
To save space, we do not repeat the above and focus our listings on the body of the \texttt{damage} function.

{\centering
	\vspace{-0.1cm}
  $\vdots$\par
  \vspace{0.1cm}
}
\noindent
Suppose we have so far implemented \texttt{damage} as follows.

{\centering
  \includegraphics[width=\linewidth]{fig/frame1.png}\par
}
\noindent
%\begin{tabular}{|p{\linewidth}}
We press keys \key{+} and \key{1}.
%\end{tabular}

{\centering
  \includegraphics[width=\linewidth]{fig/frame2.png}\par
}
\noindent
A naïve structure editor design would maintain a strict tree structure and apply
	edit commands as context-free transformations. \Hazel avoids this issue via
	Feature \feat{2}, while \Hazel's edit actions re-parse operator precedence as
	needed for typchecking. This approach is similar to MPS's side transformations \cite{GrammarCells}.

{\centering
	\vspace{-0.1cm}
  $\vdots$\par
  \vspace{0.1cm}
}
\noindent
We have in scope the player's \texttt{defenseScore}
	and want to integrate it into the damage calculation.
Our plan is to bind the current expression to a new variable \texttt{attackScore}
	and return a damage score in terms of \texttt{attackScore} and \texttt{defenseScore}.

{\centering
  \includegraphics[width=\linewidth]{fig/frame3.png}\par
}
\noindent
%\begin{tabular}{|p{\linewidth}}
We have moved the cursor to the start of the \texttt{case} expression.
Note that the dark green text cursor is augmented with a light green shading of the
	the \texttt{case} node and outlines of its child nodes [Feature \feat{3}].
	
We press \key{Enter} to create a new empty line.
Explicit encoding of line items allows us to create space around existing nodes
	in preparation for a complex node construction, just like in a text editor.
%\end{tabular}

{\centering
  \includegraphics[width=\linewidth]{fig/frame4.png}\par
}
\noindent
%\begin{tabular}{|p{\linewidth}}
We type the keys \key{l} \key{e} \key{t} \key{Space} to construct a new let line.
%\end{tabular}

{\centering
  \includegraphics[width=\linewidth]{fig/frame5.png}\par
}
\noindent
By recognizing keywords that prefix syntactic forms, \Hazel eliminates the requirement
	to learn keyboard shortcuts.

{\centering
	\vspace{-0.1cm}
  $\vdots$\par
  \vspace{0.1cm}
}
\noindent

{\centering
  \includegraphics[width=\linewidth]{fig/frame6.png}\par
}
\noindent
Now that we have created the variable \texttt{attackScore}, we want to bind it
	to the \texttt{case} expression on the following line.
In a text editor, we would delete the delimiter \texttt{in} and retype it after
	the \texttt{case} expression. Similarly, we hit \key{Backspace}.

{\centering
  \includegraphics[width=\linewidth]{fig/frame7.png}\par
}
\noindent
We have entered \emph{node staging mode} [Feature \feat{4}].
Just as a code completion menu facilitates exploration of valid token completions,
	node staging mode facilitates exploration of valid placements of a node's
	syntactic delimiters.
The \texttt{in} delimiter is highlighted in dark green to indicate that it is
	the delimiter to be placed, while the dark green guide on the left signifies
	possible positions.
The two children nodes of the let line are highlighted to indicate that, if
	we were press \key{Backspace} again, they would be deleted as well.

We press \key{$\rightarrow$} or \key{$\downarrow$} to move the delimiter to the next position.

{\centering
  \includegraphics[width=\linewidth]{fig/frame8.png}\par
}
\noindent
We press \key{Enter} to accept this position and return to normal editing mode.

{\centering
  \includegraphics[width=\linewidth]{fig/frame9.png}\par
}
\noindent
Note the similarity in keystrokes to the text editor experience.

{\centering
	\vspace{-0.1cm}
  $\vdots$\par
  \vspace{0.1cm}
}
\noindent

{\centering
  \includegraphics[width=\linewidth]{fig/frame10.png}\par
}
\noindent
We have entered our final calculation but have forgotten to account for operator precedence.
We press \key{(} at the start of the return expression.

{\centering
  \includegraphics[width=\linewidth]{fig/frame11.png}\par
}
\noindent
In the case of parentheses, \Hazel enters node staging mode automatically if it detects
	ambiguity in intent.
We press \key{$\rightarrow$} for the next position.

{\centering
  \includegraphics[width=\linewidth]{fig/frame12.png}\par
}
\noindent
Finally we press \key{Enter} or \key{)} to accept and exit node staging. 

{\centering
  \includegraphics[width=\linewidth]{fig/frame13.png}\par
}
\noindent
Once again,
note the similarity in keystrokes to the text editor experience.

\newpage
\bibliography{refs}

\end{document}